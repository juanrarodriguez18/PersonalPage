%%%%%%%%%%%%%%%%%
% This is an sample CV template created using altacv.cls
% (v1.1.5, 1 December 2018) written by LianTze Lim (liantze@gmail.com). Now compiles with pdfLaTeX, XeLaTeX and LuaLaTeX.
%
%% It may be distributed and/or modified under the
%% conditions of the LaTeX Project Public License, either version 1.3
%% of this license or (at your option) any later version.
%% The latest version of this license is in
%%    http://www.latex-project.org/lppl.txt
%% and version 1.3 or later is part of all distributions of LaTeX
%% version 2003/12/01 or later.
%%%%%%%%%%%%%%%%

%% If you need to pass whatever options to xcolor
\PassOptionsToPackage{dvipsnames}{xcolor}

%% If you are using \orcid or academicons
%% icons, make sure you have the academicons
%% option here, and compile with XeLaTeX
%% or LuaLaTeX.
% \documentclass[10pt,a4paper,academicons]{altacv}

%% Use the "normalphoto" option if you want a normal photo instead of cropped to a circle
% \documentclass[10pt,a4paper,normalphoto]{altacv}

\documentclass[10pt,a4paper,ragged2e]{altacv}

%% AltaCV uses the fontawesome and academicon fonts
%% and packages.
%% See texdoc.net/pkg/fontawecome and http://texdoc.net/pkg/academicons for full list of symbols. You MUST compile with XeLaTeX or LuaLaTeX if you want to use academicons.

% Change the page layout if you need to
\geometry{left=1cm,right=9cm,marginparwidth=6.8cm,marginparsep=1.2cm,top=1.25cm,bottom=1.25cm}

% Change the font if you want to, depending on whether
% you're using pdflatex or xelatex/lualatex
\ifxetexorluatex
  % If using xelatex or lualatex:
  \setmainfont{Carlito}
\else
  % If using pdflatex:
  \usepackage[utf8]{inputenc}
  \usepackage[T1]{fontenc}
  \usepackage[default]{lato}
\fi

% Change the colours if you want to
\definecolor{LightBlue}{HTML}{0074D9}
\definecolor{SlateGrey}{HTML}{2E2E2E}
\definecolor{LightGrey}{HTML}{666666}
\colorlet{heading}{LightBlue}
\colorlet{accent}{LightBlue}
\colorlet{emphasis}{SlateGrey}
\colorlet{body}{LightGrey}

% Change the bullets for itemize and rating marker
% for \cvskill if you want to
\renewcommand{\itemmarker}{{\small\textbullet}}
\renewcommand{\ratingmarker}{\faCircle}

%% sample.bib contains your publications
\addbibresource{sample.bib}

\begin{document}
\name{Juan Ramón Rodríguez Rosado}
\tagline{Ingeniero del Software enfocado en el BI \newline y en el desarrollo de Aplicaciones Web}
\photo{2.8cm}{Foto_Perfil}
\personalinfo{%
  % Not all of these are required!
  % You can add your own with \printinfo{symbol}{detail}
  \email{juanrarodriguez18@gmail.com}
  \phone{606987666}
  \location{Huércal de Almería, 04230}
  \newline
  % \location{Location, COUNTRY}
  \homepage{juanrarodriguez.duckdns.org}
  % \twitter{@twitterhandle}
  \linkedin{linkedin.com/in/juanrarodriguez18}
  \github{github.com/juanrarodriguez18}
  %% You MUST add the academicons option to \documentclass, then compile with LuaLaTeX or XeLaTeX, if you want to use \orcid or other academicons commands.
  % \orcid{orcid.org/0000-0000-0000-0000}
}

%% Make the header extend all the way to the right, if you want.
\begin{fullwidth}
\makecvheader
\end{fullwidth}

%% Depending on your tastes, you may want to make fonts of itemize environments slightly smaller
% \AtBeginEnvironment{itemize}{\small}

%% Provide the file name containing the sidebar contents as an optional parameter to \cvsection.
%% You can always just use \marginpar{...} if you do
%% not need to align the top of the contents to any
%% \cvsection title in the "main" bar.
\cvsection[page1sidebar]{Experiencia}

\cvevent{Consultor BI Senior}{Viewnext}{Septiembre 2017 -- Actualidad}{Almería}
\begin{itemize}
\item Desarrollo de Informes y Cuadros de Mando con MicroStrategy y Power BI.
\item Manejo de datos y desarrollos con tecnologías como Teradata, Oracle Database y SAP BW.
\item Aporte de valor a los Cuadros de Mando de MicroStrategy mediante el uso de JavaScript y CSS.
\item Desarrollo de Componentes a medida en Power BI con TypeScript.
\item Realización de manuales y búsqueda de información sobre nuevas tecnologías de los proyectos.
\item Aseguramiento de la calidad y elaboración de la documentación en la entrega de proyectos.
\item Desarrollo de flujos de datos con SAS Data Integration, enriqueciendo la funcionalidad de los flujos con código SAS.
\item Manejo de datos y desarrollos con tecnologías como Teradata, Oracle Database y SAP BW.
\end{itemize}

\divider

\cvevent{Q\&A Tester}{Sopra Steria}{Febrero 2017 -- Agosto 2017}{Sevilla}
\begin{itemize}
\item Realización de pruebas funcionales y automatizadas. 
\item Desarrollo de proyectos con las tecnologías de Java, Maven e Hibernate para proyectos internos. 
\item Realización de documentación interna y externa para la empresa. 
\end{itemize}

\cvsection{Proyectos}

\cvevent{\normalsize DreamStill, Trabajo de Fin de Grado | dreamstillapp.herokuapp.com}
{Desarrollador Web, Octubre 2016 a Junio 2017}{}{}
\begin{itemize}
\item Proyecto desarrollado con las tecnologías, Node.js, Angular2, Base de
datos NoSQL (Firebase) y con automatización de Pruebas con
SeleniumJS.
\item Aplicación Web que permite recaudar datos de distintas APIs de sueño
y mostrarlos en gráficas y eventos en un calendario Web.
\end{itemize}

\divider

\cvevent{\normalsize Shipmee, Trabajo de Asignatura | shipmee.es}
{Desarrollador de BackEnd, Febrero a Julio 2017}{}{}
\begin{itemize}
\item Proyecto desarrollado con las tecnologías Java, Maven, Spring e
Hibernate.
\item Aplicación Web para el envío de paquetes compartido, solución para
abaratar costes al enviar paquetes.
\item Equipo de 6 personas multidisciplinares.
\end{itemize}

\end{document}
